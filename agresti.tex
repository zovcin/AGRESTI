\documentclass[twoside,titlepage,12pt,a4paper]{book}
\usepackage{multicol,fancyhdr}
\usepackage{times,mathpazo}
%\usepackage{times,mathpazo,avant,chancery,charter}
% \usepackage{mathptmx,times,mathpazo,helvet,charter}
%\usepackage{times,mathptmx,helvet,charter}
%\usepackage{cmbright}
%\usepackage{ccfonts,eulervm}
%\usepackage{mathpazo,helvet,charter}
%\usepackage{libertine}
%\usepackage[libertine]{newtxmath}
%\usepackage{mathptmx,times,helvet}
%\usepackage{lmodern}
%\usepackage{mathpazo,avant,utopia,pifont}
%\usepackage{mathptmx,utopia,helvet}
\usepackage[utf8]{inputenc}
\usepackage[T1]{fontenc}
\usepackage[serbian]{babel}
\addto{\captionsserbian}{\renewcommand{\bibname}{Literatura}}
%\addto{\captionsenglish}{\renewcommand{\bibname}{References}}
\usepackage{epic}
\usepackage[Algoritam,ruled]{algorithm}
\usepackage{algpseudocode}
\usepackage{mathtools}
%\usepackage{hyperref}
\usepackage[usenames,dvips]{color}
\usepackage[square,sort,comma,numbers]{natbib} %http://merkel.zoneo.net/Latex/natbib.php
%\bibliographystyle{}
% kpsewhich -var-value TEXMFHOME za ubacivanje MiKTeXa u texlive. Treba onda i linkovati MiKTeXov texmf
% -Pdownload35 -Ppdf -G0 "%N.dvi"
%%%mk4ht htlatex qwe.tex "xhtml,charset=utf-8,pmathml" " -cunihtf -utf8 -cvalidate"
%
% Teoreme, definicije i sl.
%
\usepackage{theorem}
{\theorembodyfont{\it}
 \newtheorem{defi}{\textsc{Definicija}}}
{\theorembodyfont{\sl}
 \newtheorem{teo}{\textsc{Teorema}}}
{\theorembodyfont{\sl}
 \newtheorem{lema}{\textsc{Lema}}}
{\theorembodyfont{\rm}
 \newtheorem{napomena}{\textsc{Napomena}}}
{\theorembodyfont{\sl}
 \newtheorem{primer}{\textsc{Primer}}}
\def\qed{\hspace*{\fill}\mbox{$\rule{1ex}{1.5ex}$}}
%
% Za cirilicne reference
%
%\def\cyr{\fontencoding{OT2}\fontfamily{wncyr}\selectfont}
%\def\cyr{\fontencoding{OT2}\fontfamily{cmr}\selectfont}
%
\usepackage{amsfonts}
%
\def\R{\mathbb{R}}
\def\Z{\mathbb{Z}}
\def\N{\mathbb{N}}
\def\C{\mathbb{C}}
\def\Q{\mathbb{Q}}
%
%
%%%%%%%%%%%%%%%%%%%%%%%%%%%%%%%%%%%%%%%%%%%%%%%%%%%%%%%%%%%%%%
%
%  Razmak izme\d{d}u pasusa
%
%%%%%%%%%%%%%%%%%%%%%%%%%%%%%%%%%%%%%%%%%%%%%%%%%%%%%%%%%%%%%%
%
\parskip 3 pt
%
%
%%%%%%%%%%%%%%%%%%%%%%%%%%%%%%%%%%%%%%%%%%%%%%%%%%%%%%%%%%%%%%
%
%  Debljina linije za crteže
%
%%%%%%%%%%%%%%%%%%%%%%%%%%%%%%%%%%%%%%%%%%%%%%%%%%%%%%%%%%%%%%
%
\setlength{\unitlength}{1mm}
%
%
%%%%%%%%%%%%%%%%%%%%%%%%%%%%%%%%%%%%%%%%%%%%%%%%%%%%%%%%%%%%%%
%
%  matematički operatori su još neki
%
%%%%%%%%%%%%%%%%%%%%%%%%%%%%%%%%%%%%%%%%%%%%%%%%%%%%%%%%%%%%%%
%
\def\tg{\mathop{\rm tg}\nolimits}
\def\arctg{\mathop{\rm arctg}\nolimits}
\def\var{\mathop{\rm var}\nolimits}
\def\cov{\mathop{\rm cov}\nolimits}
\def\Co{\textbf{C}}
%
%
%%%%%%%%%%%%%%%%%%%%%%%%%%%%%%%%%%%%%%%%%%%%%%%%%%%%%%%%%%%%%%
%
% Margine
%
%%%%%%%%%%%%%%%%%%%%%%%%%%%%%%%%%%%%%%%%%%%%%%%%%%%%%%%%%%%%%%
%
% \setlength{\textwidth}{125.2mm}
% \setlength{\textheight}{199.2mm}
% \setlength{\oddsidemargin}{0mm}
% \setlength{\evensidemargin}{34mm}
% \setlength{\topmargin}{0mm}
\usepackage[a4paper]{geometry}
% \usepackage[cam,a4,pdftex,center]{crop}
% \usepackage[a4,pdftex,center]{crop}
\geometry{%
  includeheadfoot,
  margin=2cm
}
%
%
%%%%%%%%%%%%%%%%%%%%%%%%%%%%%%%%%%%%%%%%%%%%%%%%%%%%%%%%%%%%%%
%
%  Stil sa naslovom u hedingsima
%
%%%%%%%%%%%%%%%%%%%%%%%%%%%%%%%%%%%%%%%%%%%%%%%%%%%%%%%%%%%%%%
%
\pagestyle{fancy}
%\renewcommand{\chaptermark}[1]{\markboth{\slshape #1}{}}
%\renewcommand{\sectionmark}[1]{\markright{\slshape\thesubsection\ #1}}
\renewcommand{\chaptermark}[1]{\markboth{\slshape #1}{}}
\renewcommand{\sectionmark}[1]{\markright{\slshape\thesection\ #1}}
\renewcommand{\headrulewidth}{0.4pt}
\renewcommand{\footrulewidth}{0pt}
\renewcommand{\plainheadrulewidth}{0pt}
\renewcommand{\plainfootrulewidth}{0pt}
\lhead[\thepage]{\rightmark} \rhead[\leftmark]{\thepage}
\cfoot[]{} \rfoot[]{} \addtolength{\headheight}{3pt}
%
%
%%%%%%%%%%%%%%%%%%%%%%%%%%%%%%%%%%%%%%%%%%%%%%%%%%%%%%%%%%%%%%
%
% Brojanje formula po sekcijama
%
%%%%%%%%%%%%%%%%%%%%%%%%%%%%%%%%%%%%%%%%%%%%%%%%%%%%%%%%%%%%%%
\numberwithin{equation}{chapter}
%
%
%%%%%%%%%%%%%%%%%%%%%%%%%%%%%%%%%%%%%%%%%%%%%%%%%%%%%%%%%%%%%%
%
% Brojanje formula po sekcijama
%
%%%%%%%%%%%%%%%%%%%%%%%%%%%%%%%%%%%%%%%%%%%%%%%%%%%%%%%%%%%%%%
%
%\makeatletter
%\renewcommand\theequation{\thesubsection.\arabic{equation}}
%\@addtoreset{equation}{section}
%\makeatother
%
%
%%%%%%%%%%%%%%%%%%%%%%%%%%%%%%%%%%%%%%%%%%%%%%%%%%%%%%%%%%%%%%
%
% Podesavanje izgleda section i subsection naslova
%
%%%%%%%%%%%%%%%%%%%%%%%%%%%%%%%%%%%%%%%%%%%%%%%%%%%%%%%%%%%%%%
%
\usepackage{sectsty}
\chapterfont{\sffamily\bfseries\newpage\thispagestyle{plain}}
\sectionfont{\sffamily\bfseries}
\subsectionfont{\sffamily\bfseries}
%
\usepackage{url}
%
\usepackage{listings}
\lstset{language=R}
%
\title{Složeni linearni modeli}
%
\author{Zoran Ovcin}
%
\date{Školska 2021/2022}
%
%  Početak dokumenta
%
\begin{document}
%
%   Ovo sto sledi dozvoljava da se vise floatova (slika i tabela) stavi
%   na jednu stranu nego sto je default
%
\renewcommand\floatpagefraction{.9}
\renewcommand\topfraction{.9}
\renewcommand\bottomfraction{.9}
\renewcommand\textfraction{.1}
\setcounter{totalnumber}{50} \setcounter{topnumber}{50}
\setcounter{bottomnumber}{50}
%
% Ovde cu navesti reci koje LaTeX na zna da hifenizira
%
\hyphenation{a-pro-ksi-ma-ci-ja Ov-cin pro-ble-mi pro-ble-mi-ma}
%
%
%
%   NASLOV
%
%
%   početak
%
%
%\maketitle
% \addcontentsline{toc}{chapter}{Uvod}





\markboth{Uvod}{Uvod}
\chapter{Uvod}

\end{document}

\markboth{Uvod}{Uvod}
\chapter{Uvod}

Ponavljanje gradiva iz Verovatnoće, slučajnih promenljivih i statistike.

% Ova skripta je namenjena studentima master akademskih studija 
% Matematika u tehnici na Fakultetu tehničkih nauka Univerziteta
% u Novom Sadu za predmet Složeni linearni modeli.

% Preporučuje se i za predmet Analiza kategorijalnih podataka na
% istom studijskom programu.

% Kao predznanja za uspešno praćenje ovog predmeta su potrebne osnove
% Matematičke analize i Algebre, Matričnog i vektorskog računa,
% Verovatnoće, Slučajnih promenljivih, Statistike.

% Takođe je potrebno poznavanje nekog statističkog softvera. U ovoj
% skripti ćemo primere davati u softveru R.

\section{Standardne oznake}

U tekstu će se koristiti standardne matematičke oznake.
Podsetićemo ovde na neke:
\begin{itemize}
\item $\N, \Z, \Q, \R, \C$ redom skup prirodnih, celih, racionalnih, realnih i kompleksnih brojeva
\item $\R^+$ skup pozitivnih realnih brojeva
\item $\R^{m\times n}$ matrice ${m\times n}$ realnih brojeva
\item $( \Omega, {\cal F}, P )$ je oznaka za prostor verovatnoće
nad nepraznim skupom događaja $\Omega$, gde je ${\cal F}$
sigma-polje događaja i $P$ verovatnoća.
\item Slučajne promenljive nad prostorom verovatnoće ćemo
označavati velikim slovima abecede $X, Y, \ldots$ ili stavljati
oznaku nad mala ili slova grčkog alfabeta na pr.\ $\hat{\alpha}$.
\item Realizovane vrednosti slučajnih promenljivih ćemo obeležavati
odgovarajućim malim slovima: Na primer: $y_1, y_2, \ldots$ su
realizovane vrednosti slučajnih promenljivih  $Y_1, Y_2, \ldots$.
\item Događaje u prostoru verovatnoće ćemo obeležavati $\displaystyle \{ w \in \Omega | Y ( w ) \le y \} = Y \le y$.
\item Gustinu raspodele raspodele slučajne promenljive $Y$ ćemo, najčešće,
obeležavati $f ( y )$, a Funkciju raspodele $F ( y ) = P ( \{ w \in \Omega | Y ( w ) \le y \} ) = P ( Y \le y )$.
\item Očekivanje i varijansu slučajne promenljive $Y$ ćemo obeležavati redom $E ( Y )$
i $\var\; ( Y ) $, a kovarijansu $Y_1$ i $Y_2$ sa $\cov ( Y_1, Y_2 ) = E ( ( Y_1 - E ( Y_1 ) ) \,  ( Y_2 - E ( Y_2 ) )$
\end{itemize}


\section{Prostor verovatnoće i slučajne promenljive}

Neka je $\Omega$ neprazan skup, ${\cal F}$ neka je $\sigma$-polje događaja nad $\Omega$ i $P$ verovatnoća nad ${\cal F}$. Odnosno, neka je $( \Omega, {\cal} F, P )$ prostor verovatnoće.

Preslikavanje $Y : \Omega \rightarrow \R$ je \textbf{slučajna promenljiva} ako za sve realne brojeve $y$ inverzna slika $Y \le y = Y^{-1} ( \; ( -\infty, y ] \, ) \in {\cal F}$.
Svaka slučajna promenljiva ima \textbf{funkciju raspodele} $F ( y ) = P ( Y \le y )$. Više slučajnih promenljivih može da ima raspodelu sa funkcijom raspodele $F ( y )$.

Ako raspodela zavisi od parametra ili parametara $\theta$, onda pišemo $F ( y, \theta ) = P ( Y \le y )$.

\subsection{Diskretne raspodele}

Ako je kodomen preslikavanja $Y$ najviše prebrojiv skup ${\cal R}_Y = \{ y_1, y_2, \ldots \} \subseteq \R$, onda preslikavanje koje vrednostima iz kodomena $y_i$ pridružuje verovatnoće $p_i = p ( y_i )$ nazivamo \textbf{zakon raspodele}, a za $Y$ kažemo da je \textbf{diskretna} slučajna promenljiva.

\subsection{Apsolutno neprekidne raspodele}

Ako za slučajnu promenljivu $Y$ postoji nenegativna integrabilna funkcija $f : \R \rightarrow [ 0, \infty )$ tako da je $\displaystyle F ( y ) = \int_{-\infty}^y f ( t ) \, dt$, kažemo da je $Y$ \textbf{(apsolutno) neprekidna} slučajna promenljiva, a funkciju $f$ zovemo \textbf{gustina raspodele}.

\section{Važne raspodele}

\subsection{Bernulijeva raspodela}

Slučajna promenljiva $Y$ koja uzima samo vrednosti iz skupa $\{ 0, 1 \}$ kažemo da ima \textbf{Bernulijevu raspodelu} sa parametrom $p = P ( Y = 1 ) \in ( 0, 1 )$. Pišemo $Y \sim \;\mbox{Ber} ( p )$.

Kažemo i da je $Y$ indikator događaja da je neki eksperiment uspeo, da je verovatnoća uspešnosti $p$, a nije uspeo sa verovatnoćom $P ( Y = 0 ) = 1 - p$.

\subsection{Binomna raspodela}

Slučajna promenljiva koja predstavlja broj uspešnih realizacija u $n$ nezavisnih pokušaja nekog eksperimenta čija je verovatnoća uspešnosti $p \in ( 0, 1 )$ kažemo da ima \textbf{Binomnu raspodelu}, pišemo $Y \sim B ( n, p )$.

Svaki pokušaj eksperimenta ima Bernulijevu raspodelu: $Y_i \sim \;\mbox{Ber} ( p ), \: i = 1, 2, \ldots, n$ i važi $Y = \sum_{i=1}^n Y_i$.

Zakon raspodele je \textbf{Bernulijeva shema}:
$\displaystyle P( X = k ) =
\left( \genfrac{}{}{0pt}{}{n}{k} \right) p^k \, (1 - p)^{n-k}$,
$\displaystyle k \in \{ 0, 1, \ldots, n \}$.

\subsection{Poasonova raspodela}

Zakon \textbf{Poasonove raspodele} $Y \sim P ( \lambda )$, $\lambda > 0$:
$\displaystyle P ( Y = k ) = \frac{\lambda^k}{k!} \, e^{-\lambda}$,
$\displaystyle k = 0, 1, \ldots$

\subsection{Geometrijska raspodela}

\textbf{Geometrijska raspodela} sa parametrom $p \in ( 0, 1 )$ 
u oznaci $Y \sim G ( p )$ ima zakon raspodele:
$\displaystyle P ( Y = k ) = p \, ( 1 - p )^{k-1}$,
$\displaystyle k = 1, 2, \ldots$ 

Koristi se za modelovanje broja pokušaja do prve uspešne realizacije eksperimenta u nezavisnim ponavljanjima.

\subsection{Normalna raspodela}

Kažemo da slučajna promenljiva $Y$ ima \textbf{Normalnu ili Gausovu
raspodelu}, zapisujemo  $Y \sim N ( \mu, \sigma^2 )$, ako je za $\mu \in \R$,
$\sigma \in \R^+$  njena gustina raspodele
\begin{equation}\label{eq:gauss}
f ( y, \mu, \sigma ) = \frac{1}{\sqrt{2 \, \pi } \: \sigma} \:
\exp \left( -\frac12 \left( \frac{y - \mu}{\sigma} \right)^2 \right).
\end{equation}
Ako $Y \sim N ( \mu, \sigma^2 )$, očekivanje i varijansa su
$E ( Y ) = \mu$, $\var ( Y ) = \sigma^2$.

Ako $Y \sim N ( \mu, \sigma^2 )$ onda 
$\displaystyle Z = \frac{Y - \mu}{\sigma} \sim N ( 0, 1 )$
(ima standardnu normalnu raspodelu).

Ako su slučajne promenljive $Y_k$ nezavisne i
$Y_k \sim N ( \mu_k, \sigma_k^2 )$, $k = 1, 2, \ldots, n$,
onda
$$Y = a_1 \, Y_1 + a_2 \, Y_2 + \cdots + a_n \, Y_n \sim N ( \mu, \sigma^2 ), \;\mbox{gde je}$$
$$\mu = a_1 \, \mu_1 + a_2 \, \mu_2 + \cdots + a_n \, \mu_n \;\mbox{i}\;
\sigma^2 = a_1^2 \, \sigma_1^2 + a_2^2 \, \sigma_2^2 +  \cdots + 
a_n^2 \, \sigma_n^2.$$

\subsection{Hi-kvadrat raspodela i Necentralna hi-kvadrat raspodela}

Slučajna promenljiva $Y$ ima \textbf{Hi-kvadrat raspodelu sa $n$ stepeni
slobode} ($n \in \N$) ako je njena gustina raspodele
\(\displaystyle 
f ( y, n ) =
\frac{y^{n/2-1} \, e^{-y/2}}{2^{n/2} \, \Gamma (n / 2)},\: y > 0
\).
Pišemo \(Y \sim \chi^2 ( n ) \).
Očekivanje (srednja vrednost) i varijansa su redom
$E ( Y ) = n$, $\var ( Y ) = 2 n$.

Ako $Z \sim N ( 0, 1 )$, onda $Y = Z^2 \sim \chi ( 1 )$.

Ako nezavisne slučajne promenljive
$Y_1 \sim \chi^2 ( n )$ i $Y_2 \sim \chi^2 ( m )$, onda
$Y_1 + Y_2 \sim \chi^2 ( n + m )$.

Ako nezavisne slučajne promenljive 
$Z_k \sim N ( 0, 1 )$, $k = 1, 2, \ldots, n$, onda
$\displaystyle \sum_{k = 1}^n Z_k^2 \sim \chi^2 ( n )$ i onda
$\displaystyle Y = \sum_{k = 1}^n ( Z_k + \mu_k )^2 = \sum_{k = 1}^n Z_k^2 + 2 \sum_{k = 1}^n Z_k \, \mu_k + \sum_{k = 1}^n \mu_k^2 \sim \chi^2 ( n, \lambda )$ ima \textbf{Necentralnu hi-kvadrat raspodelu} sa $n$ stepeni slobode i parametrom necentralnosti $\lambda = \sum_{k = 1}^n \mu_k^2$.

Kvantili necentralne $\chi^2 ( n, \lambda )$ raspodele su veći od kvantila centralne $\chi^2 ( n )$ raspodele. Pri tome je $E ( Y ) = n + \lambda$ i $\var ( Y ) = 2 n + 4 \lambda$.

Ako nezavisne slučajne promenljive 
$Y_k \sim \chi^2 ( n_k, \lambda_k )$, $k = 1, 2, \ldots, n$, onda
$$\sum_{k = 1}^n Y_k \sim 
\chi^2 \left( \sum_{k = 1}^n n_k, \sum_{k = 1}^n \lambda_k \right).$$ 

\subsection{Studentova raspodela}

Slučajna promenljiva $T$ ima \textbf{Studentovu raspodelu}, \(\displaystyle T \sim t ( n )\) sa $n$ stepeni slobode ako ima gustinu raspodele
\[\displaystyle f ( t, n ) =
\frac{\Gamma ( (n+1) / 2)}{\sqrt{n \pi} \;\; \Gamma ( n/2 ) \, 
\left( 1 + t^2 / n \right)^{(n+1)/2}}, \: t \in \R.\]

Drugi zapis: $\displaystyle
f ( t, n ) = \left( \sqrt{n} \: B \left( \frac12, \frac{n}{2} \right) \,
\left( 1 + \frac{t^2}{n} \right)^{(n+1)/2} \right)^{-1}$.

Ako su $Z \sim N ( 0, 1 )$ i $Y \sim \chi_n^2$ nezavisne
slučajne promenljive onda
$\displaystyle T = \frac{Z}{\sqrt{\frac{Y}{n}}} \sim t ( n )$.

\[E ( T ) = 0, \quad\quad \var ( T ) = \frac{n}{n -2},\; n > 2. \]

\subsection{Fišerova raspodela}

Slučajna promenljiva ima \textbf{Fišerovu raspodelu} sa $m$ i $n$ stepeni slobode $\displaystyle F \sim F ( m, n ),\; m, n \in \N$, ako ima gustinu\\
$$f ( t, m, n ) = 
\frac{\Gamma ( (m+n) / 2) \, m^{m/2} \, n^{n/2}}%
{\Gamma ( m/2 ) \, \Gamma ( n / 2)} \cdot
\frac{t^{m/2-1}}{( m t + n )^{(m+n)/2}}
\;\mbox{za}\; t > 0.$$

Ako \(\displaystyle F \sim F ( m, n )\), onda 
$\displaystyle E ( F ) = \frac{m}{n - 2}, n > 2, \quad
\var ( F ) = \frac{2 n^2 \, ( m + n - 2)}{m \, ( n - 2 )^2 \, ( n - 4 )}, n > 4$.

Ako su $Y_1 \sim  \chi^2 ( m )$ i $Y_2 \sim \chi^2 ( n )$ nezavisne
slučajne promenljive, onda
\(\displaystyle F = \frac{\frac{Y_1}{m}}{\frac{Y_2}{n}} \sim F ( m, n )\).

Ako $T \sim  t ( n )$ onda
\(\displaystyle F = T^2 \sim F ( 1, n )\).

Ako su $Y_1$ i $Y_2$ nezavisne slučajne promenljive i
$Y_1 \sim \chi^2 ( m, \lambda )$ i $Y_2 \sim \chi^2 ( n )$,
onda slučajna promenljiva
\(\displaystyle
F = \frac{Y_1 / m}{Y_2 / n} \sim F ( m, n, \lambda )\)
ima \textbf{Necentralnu Fišerovu raspodelu}. Kvantili necentralne Fišerove
raspodele su veći od kvantila centralne Fišerove raspodele.


\section{Slučajni vektori}

Ako su $Y_1$, $Y_2$, \ldots $Y_n$ slučajne promenljive, 
sa $\mathbf{Y} = [ Y_1, Y_2, \ldots, Y_n ]^T$ obeležavamo
\textbf{slučajni vektor} $\mathbf{Y}$.

\textbf{Funkcija raspodele slučajnog vektora $\mathbf{Y}$} za $[ y_1, y_2, \ldots, y_n ]^T \in \R^{n\times1}$ je 
\[
F ( y_1, y_2, \ldots, y_n ) = P ( Y_1 \le y_1 \cap Y_2 \le y_2 \cap \cdots \cap Y_n \le y_n).
\]

Neprekidni slučajni vektori imaju \textbf{zajedničku gustinu raspodele}: nenegativnu integrabilnu funkciju $f : \R^n \rightarrow [ 0, \infty )$ koja zadovoljava
\[
F ( y_1, y_2, \ldots, y_n ) = \int_D  f ( x_1, x_2, \ldots, x_n) \: dx_1 dx_2 \cdots dx_n, \;\mbox{gde je}
\]
$\displaystyle D = ( -\infty, y_1 ] \times  ( -\infty, y_2 ] \times  \cdots \times  ( -\infty, y_n ]$.

Raspodele komponenti slučajnog vektora $Y_i$ nazivamo \textbf{marginalne raspodele}.

Očekivanje i varijansu slučajnog vektora definišemo
$$E ( \mathbf{Y} ) = \left[
\begin{array}{c}
E ( Y_1) \\
E ( Y_2) \\
\vdots \\
E ( Y_n) \\
\end{array} \right] \;\mbox{i}\; 
\var ( \mathbf{Y} ) = \left[
\begin{array}{c}
\var ( Y_1) \\
\var ( Y_2) \\
\vdots \\
\var ( Y_n) \\
\end{array} \right].$$

Na osnovu osobina očekivanja i varijanse lako se vidi da je
za konstantnu matricu $A$, $E(A) = A$,
$\var(A) = \mathbf{0}$, $E ( A \, \mathbf{Y} ) = A \,
E ( \mathbf{Y} )$.

\textbf{Varijansno-kovarijansnu matricu} $\cov ( \mathbf{Y} )$ slučajnog vektora $\mathbf{y}$
definišemo
\[
\cov ( \mathbf{Y} ) = \left[
\begin{array}{cccc}
\var ( Y_1 )      & \cov ( Y_1, Y_2 ) &        & \cov ( Y_1, Y_n )\\
\cov ( Y_2, Y_1 ) & \var ( Y_2 )      &        & \cov ( Y_2, Y_n )\\
                  &                   & \ddots &                  \\
\cov ( Y_n, Y_1 ) & \cov ( Y_n, Y_2 ) &        & \var ( Y_n )     \\
\end{array}
\right].
\]
Poznato je $\cov ( Y_i, Y_i ) = \var ( Y_i )$ i 
$\cov ( Y_i, Y_j ) = \cov ( Y_j, Y_i )$.
Matrica $\cov ( \mathbf{Y} )$ je simetrična, odnosno
$\cov ( \mathbf{Y} ) = \cov ( \mathbf{Y} )^T$.
Za konstantnu matricu $A$ važi
$\displaystyle \cov ( A \, \mathbf{Y} ) = A \, \cov ( \mathbf{Y} ) \, A^T$.

\subsection{Višedimenzionalna normalna raspodela}

Ako komponente slučajnog vektora $\mathbf{Y} = [ Y_1, Y_2, \ldots, Y_n ]^T$ imaju normalnu raspodelu, $Y_i \sim N ( \mu_i, \sigma_i^2 ), \: i = 1, 2, \ldots, n$, 
kažemo da $\mathbf{Y}$ ima \textbf{Višedimenzionalnu normalnu raspodelu},
$\mathbf{Y} \sim MVN ( \boldsymbol{\mu}, V )$, gde je $\boldsymbol{\mu} = [ \mu_1, \mu_2, \ldots, \mu_n ]^T$, $V = \cov ( \mathbf{y} )$.

Ako $\mathbf{Y} \sim MVN ( \boldsymbol{\mu}, V )$ onda:
\begin{itemize}
\item Matrica $V$ je simetrična i pozitivno semidefinitna.
\item U $i$-toj vrsti i $j$-toj koloni matrice $V$ je $\rho_{i,j} \, \sigma_i \, \sigma_j$, gde je $\rho_{i,j}$ \textbf{koeficijent korelacije} $Y_i$ i $Y_j$.
\item Dijagonalni elementi matrice $V$ su $\sigma_i^2$.
% \item Linearna kombinacija komponenti slučajnog vektora $\mathbf{Y}$ ima Normalnu raspodelu.
\item Ako $V$ ima inverznu matricu $V^{-1}$ onda
\[
( \mathbf{Y} - \boldsymbol{\mu} )^T \, V^{-1} \, ( \mathbf{Y} - \boldsymbol{\mu} )
\sim \chi^2 ( n ) \;\;\mbox{i}\;\;\;
\mathbf{Y}^T \, V^{-1} \, \mathbf{Y} \sim \chi^2 ( n, \lambda ),
\;\mbox{gde je}\; \lambda = \boldsymbol{\mu}^T \, V^{-1} \, \boldsymbol{\mu}.
\]
\item Ako $V$ ima inverznu matricu $V^{-1}$ onda je gustina slučajnog vektora $\mathbf{y}$:
$$\displaystyle f ( y, \boldsymbol{\mu}, V ) =  \det(2 \, \pi \, V)^{-{\frac{1}{2}}} \, \exp\left(-{\frac{1}{2}}(y - \boldsymbol{\mu})^{T} \, V^{-1} \, (y - \boldsymbol{\mu})\right),$$
za $y = [ y_1, y_2, \ldots, y_n ]^T$.
\end{itemize}


\chapter{Statističko zaključivanje}

U ovom poglavlju ćemo dati neke pojmove, formule i teoreme iz Statistike. Preskočićemo oblasti deskriptivne statistike i ponoviti neke delove statističkog zaključivanja.

\section{Statistika, osnovni pojmovi}

\textbf{Populacija} je skup svih elemenata koje ispitujemo.
\textbf{Obeležje} je numerička karakteristika elementa. Modeliramo
ga slučajnom promenljivom.
\textbf{Uzorak} je odabrani deo populacije na kojem ispitujemo
realizovanu vrednost obeležja $X$.

\textbf{Prost slučajni uzorak} ili, kraće, \textbf{uzorak}, je $n$-dimenzionalni slučajni vektor čije komponente su nezavisne i imaju raspodelu posmatranog obeležja $\mathbf{Y} = [ Y_1, Y_2, \ldots, Y_n ]^T$. 

Komponente slučajnog vektora $\mathbf{Y}$ su IID = Independent, Identicaly Distributed (EN).

\textbf{Realizovana vrednost prostog slučajnog uzorka} se obeležava malim slovima:\\[3pt]
$[ Y_1, Y_2, \ldots, Y_n ]^T \rightarrow \mathbf{y} = [ y_1, y_2, \ldots, y_n ]^T$.

\textbf{Statistika} je slučajna promenljiva koja nastaje kao funkcija uzorka.
Realizovanu vrednost statistike obeležavamo odgovarajućim malim slovom.


\section{Važne statistike uzorka $[Y_1, Y_2, \ldots, Y_n]^T$ obeležja $Y$}

\subsection{Srednja vrednost uzorka}
$\displaystyle \bar{Y}_n = \frac1n \sum_{k = 1}^n Y_k$

\[
E ( \bar{Y}_n ) = E ( Y ), \quad\quad
\var ( \bar{Y}_n ) = \frac1n \, \var ( Y ).
\]

Ako $\displaystyle Y \sim N ( \mu, \sigma^2 ) \; \mbox{onda} \; \bar{Y}_n \sim N \left( \mu, \frac{\sigma^2}{n} \right)$ i $\displaystyle \sum_{k = 1}^n \left( \frac{Y_k - \mu}{\sigma} \right)^2 \sim \chi_n^2$.

\subsection{Uzorački momenti}

$\displaystyle M_r = \frac1n \sum_{k = 1}^n Y_k^r$, \textbf{momenat reda $r$}\\[3pt]
\indent $\displaystyle \hat{\mu}_r = \frac1n \sum_{k = 1}^n ( Y_k - \bar{Y}_n )^r, \textbf{centralni momenat reda $r$}$.

\subsection{Uzoračka varijansa}

Centralni momenat reda dva
$\displaystyle \hat{\mu}_2 = \frac1n \sum_{k = 1}^n ( Y_k - \bar{Y}_n )^2
= \frac1n \sum_{k = 1}^n Y_k^2  - \bar{Y}_n^2 = M_2 - M_1^2$
nije centriran već samo asimptotski centriran ocenjivač varijanse obeležja:
\(\displaystyle
E ( \hat{\mu}_2 ) = \frac{n - 1}{n} \, \var ( Y )
\).
Stoga \textbf{uzoračku varijansu} definišemo:
$\displaystyle S^2 = \frac{n}{n - 1} \, \hat{\mu}_2 = \frac{1}{n - 1} \sum_{k = 1}^n ( Y_k - \bar{Y}_n )^2$, a \textbf{uzoračku standardnu devijaciju} $S = \sqrt{S^2}$.

Ako $\displaystyle Y \sim N ( \mu, \sigma^2 )$ onda $\displaystyle \frac{( n - 1 ) \, S^2}{\sigma^2} = \sum_{k = 1}^n \left( \frac{Y_k - \bar{Y}_n}{\sigma} \right)^2 \sim \chi_{n - 1}^2$.


\section{Tačkaste ocene parametara raspodele obeležja $Y$ za uzorak $[Y_1, Y_2, \ldots, Y_n]^T$}

Raspodela obeležja zavisi od (nepoznatog) parametra $\theta$, koga ocenjujemo pomoću (realizovane vrednosti) uzorka.

\textbf{Ocenjivač} neke funkcija parametra $\tau ( \theta )$ je statistika $\displaystyle U = u ( X_1, X_2, \ldots, X_n )$ čija realizovana vrednost (\textbf{ocena}) $\displaystyle u ( x_1, x_2, \ldots, x_n )$ je bliska $\tau ( \theta )$.

Ocenjivač $U$ je \textbf{centriran} za $\tau ( \theta )$ ako 
$\displaystyle E ( U ) = \tau ( \theta )$, odnosno, \textbf{asimptotski centriran} ako
$\displaystyle \lim\limits_{n \rightarrow \infty} E ( U ) = \tau ( \theta )$.

Ocenjivač $U$ je \textbf{postojan} za $\tau ( \theta )$ ako 
$\displaystyle \lim\limits_{n \rightarrow \infty}
P ( | \tau ( \theta ) - u ( X_1, X_2, \ldots, X_n ) | > \varepsilon ) = 0$
za sve $\varepsilon > 0$.

Koristeći nejednakost Čebiševa za centrirane ocenjivače sa konačnom varijansom se može pokazati da su postojani.

\noindent\textbf{Srednja kvadratna greška} ocenjivača $U$ za $\tau ( \theta )$ je
\mbox{$E ( ( U - \tau (\theta ) )^2 ) =
D ( U ) + ( ( E ( U ) - \tau ( \theta ) )^2$.}

Ako su $U_1$ i $U_2$ centrirani ocenjivači za $\tau ( \theta )$
i $D ( U_1 ) < D ( U_2 )$, kažemo da je ocenjivač $U_1$
\textbf{efikasniji} od $U_2$. Za obeležje i parametar postoji
\textbf{najbolja} disperzija $\sigma_0^2$ koja se može postići.
$\sigma_0^2$ se može oceniti koristeći nejednakost Rao-Kramer.

\subsection{Metod momenata}

Ovom metodom ocene parametara dobijamo iz jednačina u kojima izjednačavamo 
uzoračke momenta sa momentima obeležja.

Na primer, za Normalnu raspodelu $Y \sim N ( \mu, \sigma^2 )$, iz jednačina
$E ( Y ) = M_1 = \bar{Y}_n$ i $\var Y = \hat{\mu}_2$, dobijamo
$\mu = \bar{Y}_n = \frac1n \sum_{k=1}^n Y_k$ i $\sigma = \sqrt{\sigma^2} = \sqrt{\frac1n \sum_{k=1}^n ( Y_k - \bar{Y}_n )^2}$.


\subsection{Metod maksimalne verodostojnosti}

Ako je poznata raspodela obeležja $Y$ koja zavisi od nepoznatog parametra $\theta$ i ako je poznata realizovana vrednost uzorka $\mathbf{y}$, možemo potražiti vrednost paratetra $\theta$ za koju je dobijeni uzorak najverovatniji. Dobijena formula daje ocenjivač \textbf{maksimalne verodostojnosti} (MLE = Maximum Likelihood Estimate).

Za ocenu parametra $\theta$ od koga zavisi gustina raspodele $f ( y, \theta )$ ili zakon raspodele $p_i = p ( y_i, \theta )$ uzima se vrednost $\theta = \theta ( y_1, y_2, \ldots, y_n )$ za koju se ostvaruje maksimum \textbf{funkcije verodostojnosti} koja se za realizovanu vrednost uzorka $( y_1, y_2, \ldots, y_n)$ računa:\\[3pt]
\[
L = L ( y_1, y_2, \ldots, y_n, \theta ) = 
\left\{ 
\begin{array}{rl}
\varphi ( y_1, \theta ) \, \varphi ( y_2, \theta ) \, \ldots \, \varphi ( y_n, \theta ), & \mbox{za neprekidno}\\
p ( y_1, \theta ) \, p ( y_2, \theta ) \, \ldots \, p ( y_n, \theta ), & \mbox{za diskretno obeležje.}
\end{array}
\right.
\]

Ovaj postupak se može primeniti i ako se ocenjuje više parametara, odnosno, ako je $\theta$ vektor nepoznatih parametara raspodele. U mnogim slučajevima ne postoji zatvorena formula koja daje ocenjivače, već se do ocene maksimalne verodostojnosti dolazi primenom postupaka nelinearne numeričke optimizacije.

Na primer, ako za uzorak sa Normalnom raspodelom $Y \sim N ( \mu, \sigma^2 )$ primenjujemo MLE za ocenu parametara $\mu$ i $\sigma^2$, koristimo teoriju funkcija više promenljivih za nalaženje maksimuma funkcije dve promenljive.
Dobijamo za ocenu $\mu$ i $\sigma^2$ iste formule kao metodom momenata:
$\mu = \bar{Y}_n = \frac1n \sum_{k=1}^n Y_k$ i $\sigma^2 = \hat{\mu}_2 = \frac1n \sum_{k=1}^n ( Y_k - \bar{Y}_n )^2$.

\section{Intervali poverenja za parametar raspodele obeležja $Y$}

Koristimo uzorak $[Y_1, Y_2, \ldots, Y_n]^T$, odnosno, realizovanu vrednost $[y_1, y_2, \ldots, y_n]^T$.

Za obeležje $Y$ čija je funkcija raspodele $F ( y, \theta )$, ako su $U_1 = u_1( Y_1, Y_2, \ldots, Y_n )$ i $U_2 = u_2( Y_1, Y_2, \ldots, Y_n )$ statistike za koje važi \(P ( U_1 < \theta < U_2 ) = \beta\), gde je $\beta \in ( 0, 1 )$ unapred zadat \textbf{nivo poverenja}, kažemo da je $(U_1, U_2)$ \textbf{interval poverenja} širine $\beta$ nepoznatog parametra $\theta$.

\subsection{Za očekivanje $\mu$ obeležja $Y \sim N ( \mu, \sigma^2 )$}

Neka su $\bar{Y}_n$ uzoračka srednja vrednost, $S^2$ uzoračka varijansa i $S = \sqrt{S^2}$.

Onda $\displaystyle T = \frac{\bar{Y}_n - \mu}{S} \, \sqrt{n} \sim t_{n-1}$.

Označimo $\left( \frac{1+\beta}{2} \right)$-kvantil raspodele $t_{n-1}$ sa $t_{(1+\beta)/2; n-1}$, t.j.\ \(\displaystyle P \left( \left| T \right| < t_{(1+\beta)/2; n - 1} \right) = \beta\). 

Statistike $U_1, U_2$ su:
$$ U_1 = \bar{Y}_n - t_{(1+\beta)/2; n - 1} \, \frac{S}{\sqrt{n}},\quad  U_2 = \bar{Y}_n + t_{(1+\beta)/2; n - 1} \, \frac{S}{\sqrt{n}}.$$

Vrednost $\displaystyle \frac{S}{\sqrt{n}} = \frac{\sqrt{\hat{\mu}_2}}{\sqrt{n-1}}$ nazivamo \textbf{standardna greška} (SE, Standard Error, EN).


\section{Statistički testovi}

Statistički testovi su postupak statističkog zaključivanja u kome se na osnovu uzorka donosi zaključak o odbacivanju neke hipoteze $H_0$ (\textbf{nulta hipoteza}) o raspodeli obeležja. Pri odbacivanju se može pogrešiti, ali se verovatnoća greške ograničava \textbf{pragom značajnosti} $\alpha$.

Nultoj hipotezi se suprotstavlja \textbf{alternativna hipoteza} $H_1$ koja je obično komplement (suprotna) od nulte hipoteze. Odbacivanje nulte hipoteze implicira usvajanje alternativne, pri čemu se takođe može pogrešiti.

\begin{tabular}{l}
Hipoteza $H_0$ protiv $H_1$\\[8pt]
Tabela mogućih grešaka
\end{tabular}
\hfill
\begin{tabular}{c|c|c}
       &  Usvojena $H_0$ & Usvojena $H_1$ \\\hline
Tačna $H_0$  &      OK         & Greška I vrste \\\hline
Tačna $H_1$  & Greška II vrste &      OK        \\
\end{tabular}\hfill\makebox{}\\

Parametarski testovi se koriste u medicini i drugim naukama za testiranja "{}ispravnosti"{} uzorka. U tom kontekstu se greška prve vrste još zove \textbf{lažno pozitivan}, a greška druge vrste \textbf{lažno negativan} rezultat.

Komplement verovatnoće greške druge vrste se zove \textbf{snaga testa}, odnosno, ako je $\beta$ verovanoća greške druge vrste, onda je $1 - \beta$ \textbf{snaga testa}.

Snaga testa može da se oceni, a ocena raste sa veličinom uzorka. Često se za prag značajnosti uzima $\alpha = 5\%$, a za snagu testa zahteva da bude barem $\beta = 80\%$.

\subsection{Parametarske hipoteze}

Postupak testiranja parametarskih hipoteza:
\begin{itemize}
\item Zadaje se prag značajnosti $\alpha$ (recimo $\alpha = 5\% = 0.05$).
\item Bira se parametar raspodele obeležja $\theta$ i ocenjivač $\hat{\theta}$ za uzorak $[Y_1, Y_2, \ldots, Y_n]^T$.
\item Nalazi se kritična oblast $C$ parametra, takva da je $P_{{H_0}} ( \hat{\theta} \in C ) = \alpha$. $C$ sadrži vrednosti ocene čija je verovatnoća mala, ukupno manja od $\alpha$, ako je $H_0$ tačna. (Vrednosti za koje mislimo da nisu plod slučajnosti.)
\item Za realizovanu vrednost uzorka se računa statistika $\theta$ i ako $\theta \in C$, odbacujemo $H_0$ (i usvajamo $H_1$).
\item Poželjno je izračunati p-vrednost (p-value, EN) $\alpha^*$ kao verovatnoću da se za ocenu dobije izračunata vrednost ili još ekstremnija.
\end{itemize}


\newpage
\subsubsection{Testiranje $H_0 ( \mu = \mu_0 )$ protiv $H_1 ( \mu \neq \mu_0 )$ za obeležje $Y \sim N ( \mu, \sigma^2 )$}

\begin{multicols}{2}
Statistika koja se koristi
\[\displaystyle T := \frac{\bar{Y}_n - \mu_0}{S}{\sqrt{n}}  \sim t_{n-1}, \]
ima Studentovu raspodelu sa $n-1$ step.\ slob.\ (gde je $S^2$ uzoračka varijansa i $S = \sqrt{S^2}$).

Ako je $H_0$ tačna, očekivana vrednost $E ( T ) = 0$, za kritičnu oblast uzimamo vrednosti van intervala $( -t_{1-\alpha/2;n-1}, t_{1-\alpha/2;n-1} )$, gde je 
$P ( | T | < t_{1-\alpha/2;n-1} ) = 1 - \alpha$.
%$ =  P ( -t_{1-\alpha/2;n-1} < T < t_{1-\alpha/2;n-1} )$

\begin{picture}(80,50)
\put(-5,-4){\includegraphics[width=.52\textwidth]{plot_pdf_x_absle_200.pdf}}
\put(13,44){\makebox(0,0)[tl]{Gustina raspodele statistike $T$}}
%\put(13,46){\makebox(0,0)[tl]{$1 - \alpha = P ( | T | < t_{1-\alpha/2;n-1} ) =$}}
%\put(11,40){\makebox(0,0)[tl]{$= P ( -t_{1-\alpha/2;n-1} < T < t_{1-\alpha/2;n-1} )$}}
\put(14,17){\makebox(0,0)[tl]{$\frac{\alpha}{2}$}}
\put(66,17){\makebox(0,0)[tl]{$\frac{\alpha}{2}$}}
\put(35,21){\makebox(0,0)[tl]{$1 - \alpha$}}
\put(15,7){\makebox(0,0)[tl]{$-t_{1-\alpha/2;n-1}$}}
\put(55,7){\makebox(0,0)[tl]{$t_{1-\alpha/2;n-1}$}}
\end{picture}
\end{multicols}

Iskaz da realizovana vrednost statistike $T$ upada u kritičnu oblast je ekvivalentan iskazu da $\mu_0$ ne pripada intervalu poverenja za $\mu$ širine $\beta = 1 - \alpha$.
Važi $(1+\beta)/2 = 1 - \alpha/2$.
\[
\mu_0 \in \R \left\backslash
\left( \bar{y}_n \mp t_{(1+\beta)/2;n-1} \, \frac{S}{\sqrt{n}} \right)\right.
\;\Leftrightarrow\;
t := \frac{| \bar{y}_n - \mu_0 |}{S}{\sqrt{n}} > t_{1 - \alpha/2;n-1}
\;\Leftrightarrow\;
\alpha^* < \alpha,
\]
gde je $\alpha^*$ p-vrednost:
\[
\alpha^* :=  P_{H_0} \left( | T | >
\frac{| \bar{y}_n - \mu_0 |}{S}{\sqrt{n}}  \right).
\]

\subsubsection*{Jednostrani testovi - Alternativna hipoteza je $H_1 ( \mu < \mu_0 )$ ili
$H_1 ( \mu > \mu_0 )$ }

\begin{multicols}{2}


Ako je $H_0$ tačna i iz prirode obeležja se zna da srednja vrednost ne može biti manja od $\mu_0$, alternativna hipoteza je $H_1 ( \mu > \mu_0 )$.

Kritična oblast za realizovanu vrednost statistike $t := \frac{\bar{y}_n - \mu_0}{S}{\sqrt{n}}$ je $C = ( t_{1- \alpha;n-1} , \infty  )$, gde je $t_{1- \alpha;n-1}$ kvantil $1-\alpha$ raspodele $t_{n-1}$.

P-vrednost je \(\alpha^* :=  P_{H_0} ( T > t )\).

Važi i $\displaystyle t \in C \;\Leftrightarrow\; \alpha^* < \alpha$.

Analogno se testira za $H_1 ( \mu < \mu_0 )$.

\begin{picture}(80,50)
\put(-5,-4){\includegraphics[width=.52\textwidth]{plot_pdf_x_ge_100.pdf}}
\put(27,44){\makebox(0,0)[tl]{$H_1 ( \mu > \mu_0 )$}}
\put(63,18){\makebox(0,0)[tl]{$\alpha$}}
\put(34,21){\makebox(0,0)[tl]{$1 - \alpha$}}
\put(47,7){\makebox(0,0)[tl]{$t_{1-\alpha;n-1}$}}
\end{picture}
\end{multicols}

\subsection{Testiranje jednakost srednjih vrednosti dva uzorka sa Normalnom raspodelom, T-test}

Pretpostavimo da su $Y^1 \sim N ( \mu_1, \sigma_1^2 ) \;\mbox{i}\; Y^2 \sim N ( \mu_2, \sigma_2^2 )$ posmatrana obeležja.

Neka su uzorci posmatranih obeležja redom $[ Y_1^1, Y_2^1, \ldots, Y_{n_1}^1 ]^T$ i $[ Y_1^2, Y_2^2, \ldots, Y_{n_2}^2 ]^T$.

Ovo je takozvani \textbf{T-test}, često korišten statistički test. Testira se nulta hipoteza da su srednje vrednosti obeležja u dva uzorka jednake, $H_0 ( \mu_1 = \mu_2 )$.

Alternativna hipoteza je $H_1 ( \mu_1 \neq \mu_2 )$, a mogu biti, kao sa jednim uzorkom, jednostrane alternativne hipoteze: $H_1 ( \mu_1 > \mu_2 )$ ili $H_1 ( \mu_1 < \mu_2 )$

Pod pretpostavkom da su varijanse uzoraka jednake: $\sigma_1^2 = \sigma_2^2$, koristimo statistiku
\[
T := \left. \left( \bar{Y}_{n_1}^1 - \bar{Y}_{n_2}^2 - \mu_1 + \mu_2 \right) \middle/ \sqrt{S^2 \left( \frac{1}{n_1} + \frac{1}{n_2} \right) } \right., \;\mbox{gde su}\;
\bar{Y}_{n_1}^1 = \frac1n \sum_{k=1}^{n_1} Y_k^1,\: \bar{Y}_{n_2}^2 = \frac1n \sum_{k=1}^{n_2} Y_k^2 \;\;\mbox{i} 
\]
\[ S^2 = \left. \left( \sum_{k=1}^{n_1} ( Y_k^1 - \bar{Y}_{n_1}^1 )^2 + \sum_{k=1}^{n_2} ( Y_k^2 - \bar{Y}_{n_2}^2 )^2 \right) \middle/ ( n_1 + n_2 - 2) \right..
\]
Statistika $T$ ima Studentovu raspodelu sa $n_1 + n_2 - 2$ stepeni slobode $t_{n_1+n_2-2}$.

Ako postoji značajna razlika varijansi dva uzorka, koristimo statistiku
\[
T := \left. \left( \bar{Y}_{n_1}^1 - \bar{Y}_{n_2}^2 - \mu_1 + \mu_2 \right) \middle/ \sqrt{\frac{\bar{S}_1^{2}}{n_1} + \frac{\bar{S}_2^{2}}{n_2}} \right., \;\mbox{gde su}\;
\bar{S}_1^{2} \;\mbox{i}\; \bar{S}_2^{2} \;\mbox{redom uzoračke varijanse.}
\]
Statistika $T$ približno približno ima Studentovu raspodelu $t_{\nu}$ gde se za ${\nu}$ uzima procena Welcha:
\[\nu := \frac{\left( \frac{\bar{s}_1^{2}}{n_1} + \frac{\bar{s}_2^{2}}{n_2} \right)^2}{\frac{1}{n_1 - 1} \left( \frac{\bar{s}_1^{2}}{n_1} \right)^2 + \frac{1}{n_2 - 1} \left( \frac{\bar{s}_2^{2}}{n_2} \right)^2}, \;\;\mbox{računata sa realizovanim vrednostima varijansi.}\]
Pod pretpostavkom da je $H_0$ tačna, kao i kod testova sa jednim uzorkom, koristimo kvantile odgovarajuće Studentove raspodele za nalaženje kritične oblasti.

\subsubsection{T-test parova}

\textbf{T-test parova} se koristi kad imamo dva obeležja čije su vrednosti uparene kao da su uzimane "{}pre"{} i "{}posle"{} nekog tretmana. Neka su to redom realizovane vrednosti $[ y_1^1, y_2^1, \ldots, y_{n}^1 ]^T$ i $[ y_1^2, y_2^2, \ldots, y_{n}^2 ]^T$. Uzorci mogu biti određeni diskretnim atributom koji ima binarnu vrednost.

Test jednakosti srednjih vednosti se svodi na kreiranje uzorka razlika i testiranje da li je njegova srednja vrednost jednaka nuli.

Računamo $t_1 = y_1^1 - y_1^2$, $t_2 = y_2^1 - y_2^2$, $\ldots$, $t_n = y_n^1 - y_n^2$ i testiramo $H_0 ( \mu = 0 )$ protiv $H_1 ( \mu \neq 0 )$ ili $H_1 ( \mu < 0 )$ ili $H_1 ( \mu > 0 )$, za realizovani uzorak $[ t_1, t_2, \ldots, t_n ]^T$.

\subsection{Statistički softver R i komanda \texttt{t.test}}

Svi do sad pomenuti parametarski testovi i intervali poverenja su u softveru R objedinjeni u komandu \texttt{t.test}.

Na primer:
\begin{itemize}
\item
Testiranje jednakosti srednjih vrednosti dva uzorka sa nejednakim varijansama koji su smešteni u promenljive \texttt{y1} i \texttt{y2} (uz procenu Welcha za $\nu$) se vrši komandom:\\[2pt]
\texttt{t.test(y1, y2)}
\item
Ako želimo da postavimo alternativnu hipotezu $H_1 ( y_1 > y_2 )$, dodajemo \texttt{alternative = c("greater")} kao opciju, (za obrnuto \texttt{alternative = c("less")}):\\[2pt]
\texttt{t.test(y1, y2, alternative = c("greater"))}
\item
Ako smatramo da su varijanse uzoraka jednake, dodajemo opciju \texttt{var.equal = TRUE}:\\[2pt]
\texttt{t.test(y1, y2, var.equal = TRUE)}
\item
Ako su uzorci uparenih vrednosti, za t-test parova dodajemo opciju \texttt{paired = TRUE}:\\[2pt]
\texttt{t.test(y1, y2, paired = TRUE)}
\item
Za testiranje srednje vrednosti jednog uzorka se zadaje \texttt{broj} za vrednost $\mu_0$ opcijom \texttt{mu = broj}:\\[2pt]
\texttt{t.test(y1, mu = broj)}
\item
Softver će vratiti p-vrednost \texttt{p.value}, vrednost $t$ statistike \texttt{statistic}, $95\%$ interval poverenja koji odgovara pragu značajnosti $\alpha = 0.05 = 5\%$ i ostale podatke vezane za testiranje.
\item
Ako, na primer, za testiranje srednje vrednosti jednog uzorka želimo umesto $95\%$ neki drugi nivo poverenja \texttt{nivo}, zadajemo opciju \texttt{conf.level = nivo}:\\[2pt]
\texttt{t.test(y1, mu = broj, conf.level = nivo)}
\end{itemize}
\end{document}


% \subsection{Testiranje jednakosti proporcija za dva uzorka}

% % Da li veruju u zagrobni život? Pitali su 684 žena, 550 odgovorilo sa DA i 563 muškarca, 425 odgovorilo sa DA. Testirati hipotezu da su proporcije jednake.

% $H_0 ( p_1 = p_2 )$ protiv $H_1 ( p_1 \neq p_2 )$

% Pretpostavljamo da broj pozitivnih odgovora žena ima
% $X_1 : {\cal B} ( n_1, p_1 )$
% a muškaraca $X_2 : {\cal B} ( n_2, p_2 )$.

% Neka $\hat{p} = \frac{X_1 + X_2}{n_1 + n_2}$, onda statistika
% $\displaystyle Z = \frac{\hat{p}_1 - \hat{p}_2}%
% {\sqrt{\hat{p} \, ( 1 - \hat{p} ) \, ( 1 / n_1 + 1 / n_2 )}}$,\\
% ako je tačna Nulta hipoteza, ima približno Normalnu
% ${\cal N} ( 0, 1 )$ raspodelu.


% \subsection{Neparametarski testovi
% $H_0 ( F ( x ) = F_0 ( x ) )$
% protiv $H_1 ( F ( x ) \neq F_0 ( x ) )$}

% \subsubsection{$\chi^2$ test}

% Uzorak se grupiše u intervale  $I_i$, sa deobenim tačkama $m_i$,
% $i = 0, 1, \ldots k$ i brojem elemenata uzorka u intervalu $i$
% jednak $f_i$, $i = 1, 2, \ldots k$. (Treba $f_i \ge 5$.)

% Može se pokazati da se za dovoljno veliki obim uzorka $n$,
% raspodela statistike 
% \[
% Y = \sum\limits_{i = 1}^k \frac{( F_i - n \, p_i )^2}{n \, p_i},
% \;\mbox{ gde je }\; p_i = P ( m_{i-1} < X \le m_i ),\:
% f_i \;\mbox{ realizovana vrednost }\; F_i,
% \]
% može aproksimirati $\chi^2_{k - 1}$ raspodelom. Ako se ocenjuje
% $s$ parametara, onda $\chi^2_{k - 1 - s}$.

% Ako realizovana vrednost statistike $y > y_{1 - \alpha}$,
% gde je $y_{1 - \alpha}$ kvantil $\chi^2$ raspodele
% sa $k - 1 - s$ stepeni slobode, $s =$ broj ocenjivanih parametara,
% odbacujemo nultu hoipotezu $H_0$.

% \subsubsection{Tabela kontigencije}

% $\chi^2$-test nezavisnosti obeležja. Obeležje $X$ uzima 
% $m$ mogućih vrednosti, $Y$ uzima $n$ mogućih vrednosti.

% Formira se tabela $m \times n$ verovatnoća izračunatih preko
% marginalnih verovatnoća $p_{i,j} = p_{i \cdot} \, p_{\cdot j}$,
% koje se dobijaju koristeći marginalne frekvencije.

% Statistika $\displaystyle Y = \sum\limits_{i, j} 
% \frac{( F_{i,j} - n \, p_{i \cdot} \, p_{\cdot j} )^2}%
% {n \, p_{i \cdot} \, p_{\cdot j}}$
% ima približno $\chi^2$ raspodelu sa $(m - 1) \, (n-1)$ stepeni
% slobode.

% \begin{primer}
% U tabeli su dati brojevi studenata koji su položili i pali
% kolokvijum kod tri asistenta. Testirati hipotezu da su
% procenti položenih nezavisni od asistenta.\\[4pt]
% \centerline{\begin{tabular}{c|c|c|c|c}
%       & X   &  Y    &  Z  & \\\hline
% pali  & 50  &  47   & 56  & 153 \\
% položili & 5 & 14 & 8 & 27 \\\hline
% ukupno & 55 & 61 & 64 \\
% \end{tabular}}
% \end{primer}

% \begin{small}
% \begin{lstlisting}[frame=single]
% chisq.test(matrix(c(50,5,47,14,56,8), ncol = 3))  # p-value = 0.08873
% \end{lstlisting}
% \end{small}
% \newpage
% \subsubsection*{Test Kolmogorov-Smirnov}

% Primenjujemo ga za poznatu neprekidnu raspodelu

% Statistika koju koristimo je
% \[
% D_n = \sup\limits_x | F_n^* ( x ) - F ( x ) |, \;\mbox{ važi }\;
% P ( \sqrt{n} \, D_n \le \lambda ) \rightarrow D ( \lambda ),
% \;\mbox{ za } n \rightarrow \infty,
% \;\mbox{ gde je}
% \]
% $D ( \lambda )$ funkcija raspodele Kolmogorov-Smirnov
% čiji kvantili su $\lambda_{0.95} = 1.36$ i $\lambda_{0.99} = 1.63$.

% \begin{primer}
% Za 100 brojeva generisanih pseudo-slučajnim generatorom u
% intervalu $( 0, 1 )$ testirati da li su uniformno raspoređeni
% testom Kolmogorov-Smirnov sa pragom značajnosti $\alpha = 0.05$.
% Ponoviti testiranje 5000 puta. Proveriti u kojem procentu
% slučajeva hipoteza biva odbačena.
% \end{primer}

% \begin{small}
% \begin{lstlisting}[frame=single]
% set.seed(12345); n<-5000; s<-numeric(n);
% for(k in 1:n){s[k]<-ks.test(runif(100),'punif')$p.value};
% sum(s<.05)/n
% \end{lstlisting}
% \end{small}

% \texttt{0.0436}


\chapter{Normalni linearni modeli}

Neka su $Y_1, Y_2, \ldots, Y_N$ nezavisne slučajne promenljive sa istom varijansom:
\begin{equation}\label{eq:linmodel}
E ( Y_i ) = \mu_i = \mathbf{X}_i^T \, \boldsymbol{\beta}, \;
Y_i \sim N ( \mu_i, \sigma^2 ),\: i = 1, 2, \ldots, N.
\end{equation}
U matričnom obliku zapis je
\begin{equation}\label{eq:linmodelmatr}
\mathbf{y} = \mathbf{X} \, \boldsymbol{\beta} + \mathbf{e},
\;\mbox{gde je}
\end{equation}
\[
\mathbf{y} = 
\begin{bmatrix}
Y_1 \\ \vdots \\ Y_N
\end{bmatrix}, \;
\mathbf{X} = 
\begin{bmatrix}
\mathbf{x}_1^T \\ \vdots \\ \mathbf{x}_N^T
\end{bmatrix}, \;
\boldsymbol{\beta} = 
\begin{bmatrix}
\beta_1 \\ \vdots \\ \beta_p
\end{bmatrix}, \;
\mathbf{e} = 
\begin{bmatrix}
e_1 \\ \vdots \\ e_N
\end{bmatrix}.
\]


\chapter{Logistička regresija}



\end{document}
